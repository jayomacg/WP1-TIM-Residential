%% 
%% Copyright 2007-2020 Elsevier Ltd
%% 
%% This file is part of the 'Elsarticle Bundle'.
%% ---------------------------------------------
%% 
%% It may be distributed under the conditions of the LaTeX Project Public
%% License, either version 1.2 of this license or (at your option) any
%% later version.  The latest version of this license is in
%%    http://www.latex-project.org/lppl.txt
%% and version 1.2 or later is part of all distributions of LaTeX
%% version 1999/12/01 or later.
%% 
%% The list of all files belonging to the 'Elsarticle Bundle' is
%% given in the file `manifest.txt'.
%% 
%% Template article for Elsevier's document class `elsarticle'
%% with harvard style bibliographic references

\documentclass[preprint,12pt,authoryear]{elsarticle}

%% Use the option review to obtain double line spacing
%% \documentclass[authoryear,preprint,review,12pt]{elsarticle}

%% Use the options 1p,twocolumn; 3p; 3p,twocolumn; 5p; or 5p,twocolumn
%% for a journal layout:
%% \documentclass[final,1p,times,authoryear]{elsarticle}
%% \documentclass[final,1p,times,twocolumn,authoryear]{elsarticle}
%% \documentclass[final,3p,times,authoryear]{elsarticle}
%% \documentclass[final,3p,times,twocolumn,authoryear]{elsarticle}
%% \documentclass[final,5p,times,authoryear]{elsarticle}
%% \documentclass[final,5p,times,twocolumn,authoryear]{elsarticle}

%% For including figures, graphicx.sty has been loaded in
%% elsarticle.cls. If you prefer to use the old commands
%% please give \usepackage{epsfig}

%% The amssymb package provides various useful mathematical symbols
\usepackage{amssymb}
%% The amsthm package provides extended theorem environments
%% \usepackage{amsthm}

%% The lineno packages adds line numbers. Start line numbering with
%% \begin{linenumbers}, end it with \end{linenumbers}. Or switch it on
%% for the whole article with \linenumbers.

\usepackage{lineno}
\usepackage{xcolor}
\usepackage{textcomp}
\usepackage{gensymb}
\usepackage{multirow}
\usepackage{makecell}
\usepackage{float}
\usepackage{longtable}
\usepackage{threeparttable}
\usepackage{biblatex}
\bibliography{references.bib} %Imports bibliography file


\journal{Journal X}

\begin{document}

\begin{frontmatter}

%% Title, authors and addresses

%% use the tnoteref command within \title for footnotes;
%% use the tnotetext command for theassociated footnote;
%% use the fnref command within \author or \affiliation for footnotes;
%% use the fntext command for theassociated footnote;
%% use the corref command within \author for corresponding author footnotes;
%% use the cortext command for theassociated footnote;
%% use the ead command for the email address,
%% and the form \ead[url] for the home page:
%% \title{Title\tnoteref{label1}}
%% \tnotetext[label1]{}
%% \author{Name\corref{cor1}\fnref{label2}}
%% \ead{email address}
%% \ead[url]{home page}
%% \fntext[label2]{}
%% \cortext[cor1]{}
%% \affiliation{organization={},
%%            addressline={}, 
%%            city={},
%%            postcode={}, 
%%            state={},
%%            country={}}
%% \fntext[label3]{}

\title{TIMES Ireland Model: Residential Sector}

\author[inst1,inst2]{Jason Mc Guire\footnote{Contact: j.mcguire@ucc.ie}}

\affiliation[inst1]{organization={Energy Policy and Modelling Group, MaREI Centre},%Department and Organization
            addressline={Environmental Research Institute}, 
            city={Cork},
            country={Ireland}}
            
\affiliation[inst2]{organization={School of Engineering},%Department and Organization
            addressline={University College Cork}, 
            city={Cork},
            country={Ireland}}


\author[inst1,inst2]{Fionn Rogan}
\author[inst1,inst2]{Olexandr Balyk}
\author[inst1,inst2]{Hannah Daly}
\author[inst1,inst2]{ \& Brian Ó Gallachóir}

\begin{abstract}
%% Text of abstract
Under the climate action bill, the Irish government have set legally binding targets to reduce greenhouse gas (GHG) emissions by 51\% in 2030, compared to 2018 and to achieve “net-zero” GHG emissions by 2050 \cite{2021Climate2021}. These legally binding targets are set to ensure Ireland complies the Paris Agreement. 
Ireland and particularly Ireland's residential sector is preforming poorly. Ireland has the lowest renewable heat of 6.3\% \cite{2021EurostatExplorer}, large and poor fabric houses is hindering progress
Heat Pumps are offering a chance to change that Heat Pump renewable energy and decarbonising electricity secotr and high COPs with large rural detached settle patterns. 
% In 2018 only 7.8% of the population of Ireland lived in flats, this is the lowest percentage in the EU-27, the second lowest was the Netherlands at 20.2% (Eurostat, 2018). The average EU-27 was 46% and the closest in comparison was the UK at 14.8% (Eurostat, 2018). Ireland has a low urbanisation percentage of 63.2%, where the European average is 74.5% and the UK is 83.4% (United Nations, Department of Economic and Social Affairs, Population Division . World Urbanization Prospects, 2018). Ireland has the 3rd largest residential living space per capita in Europe at 41.75 m2/capita, behind Denmark and Cyprus. The European average is 33.8 m2/capita (Average Floor Area per Capita, 2008). Ireland’s low urbanisation percentage has contributed to a low number of flats, and therefore a high percentage of detached, semi-detached and terrace dwellings. Ireland’s dwellings are heated primarily by individual systems sourced from fossil fuels and the large size of Ireland’s dwellings has compounded the challenge of Ireland’s settlement patterns on climate change obligations.
The Energy system optimization models (ESOMs) have been used extensively in providing insights to decision makers on issues related to climate and energy policy. 

Reducing GHG emission in the residential sector is key..
 five-year  carbon  budgets. 
 Energy  systems  optimisa-tion modelling (ESOM) is a widely-used tool to inform pathways to addresslong-term energy challenges.  This article describes a new ESOM developedto inform Ireland’s energy system decarbonisation challenge.  The TIMES-Ireland Model (TIM) is an optimisation model of the Irish energy system,which  calculates  the  cost-optimal  fuel  and  technology  mix  to  meet  futureenergy service demands in the transport, buildings, industry and agriculturesectors,  while  respecting  constraints  in  greenhouse-gas  emissions,  primaryenergy  resources and  feasible  deployment  rates.   TIM is  developed  to  takeinto account Ireland’s unique energy system context, including a very highpotential for offshore wind energy and the challenge of integrating this on arelatively isolated grid, a very ambitious decarbonisation target in the periodto 2030, the policy need to inform five-year carbon budgets to meet policytargets, and the challenge of decarbonising heat in the context of low buildingstock thermal efficiency and high reliance on fossil fuels.  To that end, modelfeatures of note include “future proofing” with flexible timeslice and spatialdefinitions, with optional hourly time resolution in electricity generation and
\end{abstract}

%%Graphical abstract
%%\begin{graphicalabstract}
%%\includegraphics{grabs}
%%\end{graphicalabstract}

%%Research highlights
%%\begin{highlights}
%%\item Research highlight 1
%%\item Research highlight 2
%%\end{highlights}

\begin{keyword}
%% keywords here, in the form: keyword \sep keyword
Energy systems optimisation model (ESOM) \sep The integrated MARKAL-EFOM system (TIMES) \sep Residential Building Stock \sep Building Efficiency sep\ Renewable Heat \sep Model description 
%% PACS codes here, in the form: \PACS code \sep code
%%\PACS 0000 \sep 1111
%% MSC codes here, in the form: \MSC code \sep code
%% or \MSC[2008] code \sep code (2000 is the default)
%% \MSC 0000 \sep 1111
\end{keyword}

\end{frontmatter}

%% \linenumbers

%% main text
\section{Sample Section Title}
\label{sec:sample1}


%Lorem ipsum dolor sit amet, consectetur adipiscing \citep{Fabioetal2013} elit, sed do eiusmod tempor incididunt ut labore et dolore magna aliqua. Ut enim ad minim veniam, quis nostrud \citet{Blondeletal2008} exercitation ullamco laboris nisi ut aliquip ex ea commodo consequat. Duis aute irure dolor in reprehenderit in voluptate velit esse cillum dolore eu fugiat nulla pariatur. Excepteur sint occaecat cupidatat non proident, sunt in culpa qui officia deserunt mollit \citep{Blondeletal2008,FabricioLiang2013} anim id est laborum.

%Lorem ipsum dolor sit amet, consectetur adipiscing elit, sed do eiusmod tempor incididunt ut labore et dolore magna aliqua. Ut enim ad minim veniam, quis nostrud exercitation ullamco laboris nisi ut aliquip ex ea commodo consequat. Duis aute irure dolor in reprehenderit in voluptate velit esse cillum dolore eu fugiat nulla pariatur. Excepteur sint occaecat cupidatat non proident, sunt in culpa qui officia deserunt mollit anim id est laborum see appendix~\ref{sec:sample:appendix}.

%% The Appendices part is started with the command \appendix;
%% appendix sections are then done as normal sections
\appendix

\section{Sample Appendix Section}
\label{sec:sample:appendix}
Lorem ipsum dolor sit amet, consectetur adipiscing elit, sed do eiusmod tempor section \ref{sec:sample1} incididunt ut labore et dolore magna aliqua. Ut enim ad minim veniam, quis nostrud exercitation ullamco laboris nisi ut aliquip ex ea commodo consequat. Duis aute irure dolor in reprehenderit in voluptate velit esse cillum dolore eu fugiat nulla pariatur. Excepteur sint occaecat cupidatat non proident, sunt in culpa qui officia deserunt mollit anim id est laborum.

%% If you have bibdatabase file and want bibtex to generate the
%% bibitems, please use
%%

\printbibliography

%% else use the following coding to input the bibitems directly in the
%% TeX file.

%%\begin{thebibliography}{00}


% %% \bibitem[Author(year)]{label}
% %% Text of bibliographic item

% \bibitem[ ()]{}

% \end{thebibliography}
\end{document}
\endinput
%%
%% End of file `elsarticle-template-harv.tex'.
